\subsection{线性方程组的一致性(Consistency Of Linear Systems)}

如果一个包含 \(n\) 个未知数的 \(m\) 个线性方程组至少有一个解,则称其为相容方程组。如果没有解,则该方程组
被称为不相容方程组。本节旨在确定给定方程组相容的条件。

对于仅包含两个或三个未知数的方程组,
说明其相容性条件很容易。两个未知数的线性方程表示二维空间中的一条直线,
而三个未知数的线性方程表示三维空间中的平面。因此,
一个包含 \(m\) 个两个未知数的线性方程组相容当且仅当这 \(m\) 个方程定义的 \(m\) 条直线至少有一个公共交点。
同样,一个包含 \(m\) 个三个未知数的方程组相容当且仅当
相关的 \(m\) 个平面至少有一个公共交点。
然而,当 \(m\) 较大时,这些几何条件可能不易直观验证,而当 \(n\) > 3 时,相交线或平面的推广则无法用肉眼直观地看到。

我们不依赖几何来建立一致性,而是使用高斯消元法。如果将相关的增广矩阵 \([\mathbf{A}|\mathbf{b}]\) 通过行运算简化为行阶梯矩阵 \([\mathbf{E}|\mathbf{c}]\),则一致性(或不一致性)变得显而易见。
假设在将 \([\mathbf{A}|\mathbf{b}]\) 简化为 \([\mathbf{E}|\mathbf{c}]\) 的过程中,出现这样一种情况:一行中唯一的非零项出现在右侧,如下所示:

\[
\text{Row }i \rightarrow 
\left(\begin{array}{cccccc|c}
* & * & * & * & * & * & * \\ 
0 & 0 & 0 & * & * & * & * \\ 
0 & 0 & 0 & 0 & 0 & 0 & \alpha \\
\bullet & \bullet & \bullet & \bullet & \bullet & \bullet & \bullet \\
\bullet & \bullet & \bullet & \bullet & \bullet & \bullet & \bullet \\
\end{array}\right)
\leftarrow \alpha \neq 0.
\]

如果这发生在第 \(i\) 行,则相关系统的第 \(i\) 个方程为

\[
0x_1 + 0x_2 + ... + 0x_n = \alpha.
\]

当 \(\alpha \neq 0\) 时,该方程无解,因此原系统也必然不一致(因为行运算不会改变解集)。
反之亦然。也就是说,如果一个系统不一致,那么在消元过程中,会出现如下形式的行

\[
\left(\begin{array}{cccccc|c}
0 & 0 & ... & 0 & \alpha
\end{array}\right), \alpha \neq 0.
\]

否则,回代过程可以完成,并产生一个解。当遇到形式为
(0 0 ... 0 \(\mid\) 0) 的行时,不会指示不一致性。这只是表示 0 = 0,
尽管这对于确定任何未知数的值没有任何帮助,但它仍然是一个真实的陈述,
因此它并不表示系统存在不一致。

还有一些其他方法可以表征系统的一致性(或不一致性)。其中一种方法是:如果增广矩阵 \([\mathbf{A}|\mathbf{b}]\) 的最后一列 \(b\) 是非基列,则最后一列不可能存在主元,因此系统是一致的,因为 (2.3.1) 的情况不会发生。
反之,如果系统是一致的,那么 (2.3.1) 的情况在高斯消元法中永远不会发生,因此最后一列不可能是基列。换句话说,\([\mathbf{A}|\mathbf{b}]\) 是一致的当且仅当 \(b\) 是非基列。

说 \(\mathbf{b}\) 是 \([\mathbf{A}|\mathbf{b}]\) 中的非基列,等同于说\([\mathbf{A}|\mathbf{b}]\) 中的所有基列都位于系数矩阵 \(\mathbf{A}\) 中。由于矩阵中基列的数量就是秩,因此一致性也可以这样来描述:
当且仅当 rank\([\mathbf{A}|\mathbf{b}]\) = rank\((\mathbf{A})\) 时,系统才是一致的。

\begin{bluebox}{线性方程组的一致性}
满足以下任意一个条件都可以说 \([\mathbf{A}|\mathbf{b}]\) 是一致的:
\begin{itemize}
    \item 行简化阶梯形矩阵 \([\mathbf{A}|\mathbf{b}]\) 中不存在形如下列的行:
    \[
    \left(\begin{array}{cccc|c}
        0 & 0 & ... & 0 & \alpha
    \end{array}\right), \alpha \neq 0
    \]
    \item \(\mathbf{b}\) 是 \([\mathbf{A}|\mathbf{b}]\) 中的非基列。
    \item rank\([\mathbf{A}|\mathbf{b}]\) = rank\((\mathbf{A})\)
    \item \(\mathbf{b}\) 是 \(\mathbf{A}\) 的基列的一个组合。
\end{itemize}
\end{bluebox}

% 红色文字
\textcolor{red}{例子2.3.1}
% 红色水平线(宽度为文本宽度,厚度0.4pt)
\color{red}\rule{\textwidth}{0.4pt}\color{black}

\textbf{问题}: 确定以下系统是否一致: 
\[
\begin{aligned} 
    x_1 + x_2 + 2x_3 + 2x_4 + x_5 &= 1, \\
    2x_1 + 2x_2 + 4x_3 + 4x_4 + 3x_5 &= 1, \\
    2x_1 + 2x_2 + 4x_3 + 4x_4 + 2x_5 &= 2, \\
    3x_1 + 5x_2 + 8x_3 + 6x_4 + 5x_5 &= 3.
\end{aligned}
\]
\textbf{解答}: 对增广矩阵 \([\mathbf{A}|\mathbf{b}]\) 应用高斯消元法,如下所示
\[
\left(\begin{array}{ccccc|c} 
\circlednum{1} & 1 & 2 & 2 & 1 & 1 \\ 
2 & 2 & 4 & 4 & 3 & 1 \\ 
2 & 2 & 4 & 4 & 2 & 2 \\ 
3 & 5 & 8 & 6 & 5 & 3 
\end{array}\right)
\to
\left(\begin{array} {ccccc|c} 
\circlednum{1} & 1 & 2 & 2 & 1 & 1 \\ 
0 & \circlednum{0} & 0 & 0 & 1 & -1 \\ 
0 & 0 & 0 & 0 & 0 & 0 \\ 
0 & 2 & 2 & 0 & 2 & 0 
\end{array}\right)
\to
\left(\begin{array}{ccccc|c} 
\circlednum{1} & 1 & 2 & 2 & 1 & 1 \\ 
0 & \circlednum{2} & 2 & 0 & 2 & 0 \\ 
0 & 0 & 0 & 0 & \circlednum{1} & -1 \\ 
0 & 0 & 0 & 0 & 0 & 0 
\end{array}\right)
\]

由于形式为 \(\left(\begin{array}{cccc|c} 0 & 0 & \dots & 0 & \alpha \end{array}\right),\ \alpha \neq 0\) 的行从未出现,
因此系统是一致的。我们还可以观察到,\(b\) 是 \([\mathbf{A}|\mathbf{b}]\) 中的非基本列,
因此 rank\([\mathbf{A}|\mathbf{b}]\) = rank \((\mathbf{A})\)。最后,通过将 \(\mathbf{A}\) 完全约化为
\(\mathbf{E_A}\),可以验证 \(\mathbf{b}\) 确实是基本列的组合,\{\(\mathbf{A_{*1}}, \mathbf{A_{*2}}, \mathbf{A_{*5}}\)\}。

% 红色文字
\textcolor{red}{练习 2.3}
% 红色水平线(宽度为文本宽度,厚度0.4pt)
\color{red}\rule{\textwidth}{0.4pt}\color{black}

\begin{enumerate}[leftmargin=*, label=\bfseries 2.3.\arabic*]

\item 确定下列哪些系统是一致的。

\begin{tabular}{@{}p{0.35\textwidth} p{0.35\textwidth}@{}}
% (a)
\begin{tabular}{@{} >{\raggedleft\arraybackslash}m{0.1em}@{\hspace{0.1em}} m{\dimexpr\linewidth-0.1em-0.1em\relax} @{} }
	{(a)} & \[\begin{aligned}
x + 2y + z &= 2, \\
2x + 4y &= 4, \\
3x + 6y + z &= 4.
\end{aligned}\]
\end{tabular}
&
% (b)
\begin{tabular}{@{} >{\raggedleft\arraybackslash}m{0.1em}@{\hspace{0.1em}} m{\dimexpr\linewidth-0.1em-0.1em\relax} @{} }
	{(b)} & \[\begin{aligned}
2x + 2y + 4z &= 0, \\
3x + 2y + 5z &= 0, \\
4x + 2y + 6z &= 0.
\end{aligned}\]
\end{tabular}
\\[0pt]
% (c)
\begin{tabular}{@{} >{\raggedleft\arraybackslash}m{0.1em}@{\hspace{0.1em}} m{\dimexpr\linewidth-0.1em-0.1em\relax} @{} }
	{(c)} & \[\begin{aligned}
x - y + z &= 1, \\
x - y - z &= 2, \\
x + y - z &= 3, \\
x + y + z &= 4.
\end{aligned}\]
\end{tabular}
&
% (d)
\begin{tabular}{@{} >{\raggedleft\arraybackslash}m{0.1em}@{\hspace{0.1em}} m{\dimexpr\linewidth-0.1em-0.1em\relax} @{} }
	{(d)} & \[\begin{aligned}
x - y + z &= 1, \\
x - y - z &= 2, \\
x + y - z &= 3, \\
x + y + z &= 2.
\end{aligned}\]
\end{tabular}
\\[0pt]
% (e)
\begin{tabular}{@{} >{\raggedleft\arraybackslash}m{0.1em}@{\hspace{0.1em}} m{\dimexpr\linewidth-0.1em-0.1em\relax} @{} }
	{(e)} & \[\begin{aligned}
2w + x + 3y + 5z &= 1, \\
4w + 4y + 8z &= 0, \\
w + x + 2y + 3z &= 0, \\
x + y + z &= 0.
\end{aligned}\]
\end{tabular}
&
% (f)
\begin{tabular}{@{} >{\raggedleft\arraybackslash}m{0.1em}@{\hspace{0.1em}} m{\dimexpr\linewidth-0.1em-0.1em\relax} @{} }
	{(f)} & \[\begin{aligned}
2w + x + 3y + 5z &= 7, \\
4w + 4y + 8z &= 8, \\
w + x + 2y + 3z &= 5, \\
x + y + z &= 3.
\end{aligned}\]
\end{tabular}
\end{tabular}

\item 构造一个 \(3 \times 4\) 矩阵 \(\mathbf{A}\) 和 \(3 \times 1\) 列 \(\mathbf{b}\) 和 \(\mathbf{c}\),使得 \([\mathbf{A}|\mathbf{b}]\) 是不相容系统的增广矩阵,而 \([\mathbf{A}|\mathbf{c}]\) 是相容系统的增广矩阵。

\item 如果 \(\mathbf{A}\) 是一个 \(m \times n\) 矩阵,秩为 \((\mathbf{A}) = m\),请解释为什么系统 \([\mathbf{A}|\mathbf{b}]\) 对于每个右侧 \(\mathbf{b}\) 都必须是相容的。

\item 方程形式为 \(y = \alpha+\beta x+\gamma x^2\) 的抛物线是否有可能经过四个点 (0, 1), (1, 3), (2, 15) 和 (3, 37)?为什么?

\item 考虑使用浮点运算(不进行缩放)来求解以下方程组:
\[
\begin{array}{l}
.835x + .667y = .168, \\
.333x + .266y = .067.
\end{array}
\]
\begin{enumerate}[label=(\alph*)]
    \item 当使用 5 位数运算时,该系统是否具有一致性?
    
    \item 使用六位数运算会发生什么情况?
\end{enumerate}

\item 为了种植某种作物,建议每平方英尺土地施用 10 单位磷、9 单位钾和 19 单位氮。假设市面上有三种品牌的肥料——\(\mathcal{X}\) 品牌、\(\mathcal{Y}\) 品牌和 \(\mathcal{Z}\) 品牌。
一磅 \(\mathcal{X}\) 品牌肥料含有 2 单位磷、3 单位钾和 5 单位氮。一磅 \(\mathcal{Y}\) 品牌肥料含有 1 单位磷、3 单位钾和 4 单位氮。一磅 \(\mathcal{Z}\) 品牌肥料只含有 1 单位磷和 1 单位氮。
请确定是否可以通过施用这三种品牌肥料的某种组合来完全满足推荐用量。

\item 假设增广矩阵 \([\mathbf{A}|\mathbf{b}]\) 通过高斯消元法简化为行阶梯形矩阵 \([\mathbf{E}|\mathbf{c}]\)。如果某一行形
\[
\left(\begin{array}{cccc|c}
    0 & 0 & ... & 0 & \alpha
\end{array}\right), \alpha \neq 0
\]
如果该行没有出现在 \([\mathbf{E}|\mathbf{c}]\) 中,那么这种形式的行是否有可能在简化过程的早期阶段出现过?为什么?

\end{enumerate}
