\subsection{2.4 齐次系统(Homogeneous Systems)}

\(n\) 个未知数的 \(m\) 个线性方程组
\[
\begin{array}{c}
    a_{11}x_1 + a_{12}x_2 + ... + a_{1n}x_n = 0, \\
    a_{21}x_1 + a_{22}x_2 + ... + a_{2n}x_n = 0, \\
    \vdots \\
    a_{m1}x_1 + a_{m2}x_2 + ... + a_{mn}x_n = 0.
\end{array}
\]

如果方程式的右侧完全由0组成,则称其为\textbf{齐次系统}。如果方程式的右侧至少有一个非零数,则该系统称为\textbf{非齐次系统}。本节旨在探讨齐次系统的一些基本概念。

在处理齐次方程组时,一致性从来都不是问题,因为无论系数值如何,
零解 \(x_1 = x_2 = ... = x_n = 0\) 始终是唯一解。下文中,
将全零组成的解称为\textbf{平凡解}。唯一的问题是:“除了平凡解之外,还有其他解吗?
如果有,我们如何才能最好地描述它们?” 和之前一样,高斯消元法可以给出答案。

使用高斯消元法将齐次系统的增广矩阵 \([\mathbf{A|0}]\) 化简为行阶梯形时,右侧的零列不会被任何三种基本行运算改变。
也就是说,任何通过行运算从 \([\mathbf{A|0}]\) 导出的行阶梯形,也必然具有 \([\mathbf{E|0}]\) 的形式。这意味着最后一列的零只是
多余的负担,没有必要在每一步都带上去。只需将系数矩阵 \(\mathbf{A}\) 化简为行阶梯形 \(\mathbf{E}\),并记住在执行回代时,右侧
完全为零。通过一个典型的例子,可以更好地理解这个过程。

为了求解齐次系统的解
\[
\begin{array}{c}
    x_1 + 2x_2 + 2x_3 + 3x_4 = 0, \\
    2x_1 + 4x_2 + x_3 + 3x_4 = 0, \\
    3x_1 + 6x_2 + x_3 + 4x_4 = 0,
\end{array}
\]
将系数矩阵简化为行阶梯形式。
\[
\mathbf{A} = 
\left(\begin{array}{cccc}
1 & 2 & 2 & 3 \\
2 & 4 & 1 & 3 \\
3 & 6 & 1 & 4
\end{array}\right)\to
\left(\begin{array}{cccc}
1 & 2 & 2 & 3 \\
0 & 0 & -3 & -3 \\
0 & 0 & -5 & -5
\end{array}\right)\to
\left(\begin{array}{cccc}
1 & 2 & 2 & 3 \\
0 & 0 & -3 & -3 \\
0 & 0 & 0 & 0
\end{array}\right)
= \mathbf{E}.
\]
因此,原始齐次系统等价于以下简化齐次系统:
\[
\begin{aligned}
    x_1 + 2x_2 + 2x_3 + 3x_4 &= 0, \\
    -3x_3 - 3x_4 &= 0.
\end{aligned}
\]
由于这个简化系统中有四个未知数,但只有两个方程,因此不可能为每个未知数求出唯一的解。我们能做的最好的事情是,
选择两个“基本”未知数——我们称之为\textbf{基本变量},并用另外两个未知数来求解它们——这两个未知数的值必须
保持任意或“自由”,因此它们被称为\textbf{自由变量}。虽然选择一组基本变量有多种可能性,但惯例是始终求解与枢轴位置对应的未知数,
或者,等价地,与基本列对应的未知数。在这个例子中,枢轴(以及基本列)位于第一和第三位置,
因此策略是应用回代法,用自由变量 \(x_2\) 和 \(x_4\) 来求解简化系统中的基本变量 \(x_1\) 和 \(x_3\)。由第二个方程得出
\[
x_3 = -x_4.
\]
代入第一个方程可得
\[
\begin{aligned}
x_1 &= -2x_2 -2x_3 - 3x_4, \\
&= -2x_2 - 2(-x_4) - 3x_4, \\
&= -2x_2 - x_4.
\end{aligned}
\]
因此,原齐次系统的所有解都可以描述为
\[
\begin{aligned}
    &x_1 = -2x_2 - x_4 \\
    &x_2 \text{ 任意}, \\
    &x_3 = -x_4, \\
    &x_4 \text{ 任意}.
\end{aligned}
\]
由于自由变量 \(x_2\) 和 \(x_4\) 取值范围涵盖所有可能值,上述表达式描述了所有可能的解。例如,当 \(x_2\) 和 \(x_4\) 取值
\(x_2 = 1\) 且 \(x_4 = -2\) 时,则有特定解
\[
x_1 = 0, x_2 = 1, x_3 = 2, x_4 = -2
\]
将来不再像上述那样描述解集,而是更方便地通过以下方式表达解集: 
\[
\left(\begin{array}{c}
x_1\\
x_2\\
x_3\\
x_4
\end{array}\right)=
\left(\begin{array}{c}
-2x_2 - x_4\\
x_2\\
-x_4\\
x_4
\end{array}\right)=
x_2\left(\begin{array}{c}
-2\\
1\\
0\\
0
\end{array}\right)+
x_4\left(\begin{array}{c}
-1\\
0\\
-1\\
1
\end{array}\right)
\]
假设 \(x_2\) 和 \(x_4\) 是自由变量,其取值范围为所有可能的数字。这种表示被称为齐次系统的\textbf{通解}。
通解的表达式强调,每个解都是两个特解的某种组合
\[
\mathbf{h}_1 = 
\left(\begin{array}{c}
-2\\
1\\
0\\
0
\end{array}\right),  
\mathbf{h}_2 =
\left(\begin{array}{c}
-1\\
0\\
-1\\
1
\end{array}\right).
\]
\(\mathbf{h}_1\) 和 \(\mathbf{h}_2\) 显然都是解,因为 \(\mathbf{h}_1\) 是在自由变量取值 \(x_2 = 1\) 和 \(x_4 = 0\) 时生成的,而
解 \(\mathbf{h}_2\) 是在 \(x_2 = 0\) 和 \(x_4 = 1\) 时生成的。

现在考虑一个由 \(m\) 个线性方程组 \([\mathbf{A|0}]\) 组成的广义齐次方程组,
其中有 \(n\) 个未知数。如果系数矩阵的 rank(\(\mathbf{A}\)) = \(r\),那么
从前面的讨论中应该可以明显看出,恰好有 \(r\) 个
基本变量(对应于 \(\mathbf{A}\) 中基本列的位置),以及
恰好有 \(n - r\) 个自由变量(对应于 \(\mathbf{A}\) 中非基本列的位置)。
使用高斯消元法将 \(\mathbf{A}\) 化为行阶梯形,
然后使用回代法根据自由变量求解基本变量,
即可得到\textbf{通解},其形式为
\[
\mathrm{x} = x_{f_1}\mathbf{h}_1 + x_{f_2}\mathbf{h}_2 + ... + x_{f_{n-r}}\mathbf{h}_{n-r},
\]
其中 \(x_{f_1}, x_{f_2}, ..., x_{f_{n-r}}\) 为自由变量,\(\mathbf{h}_1, \mathbf{h}_2, ..., \mathbf{h}_{n-r}\) 为
\(n \times 1\) 列,表示系统的特定解。由于自由变量 \(x_{f_i}\) 涵盖所有可能值,因此通解会生成所有可能的解。

通解并不依赖于所用的行阶梯形式,因为无论使用哪种行阶梯形式,使用回代法求解基本变量,都会生成一组唯一的特解,
{\(\mathbf{h}_1, \mathbf{h}_2, ..., \mathbf{h}_{n-r}\)}。无需赘述,我们可以认为这是正确的,因为使用任何行阶梯形式求解基本变量,
其结果必然与将 \(\mathbf{A}\) 完全简化为 \(\mathbf{E_A}\),然后求解简化齐次系统的基本变量,所得结果完全相同。\(\mathbf{E_A}\) 的唯一性,保证了 \(\mathbf{h}_i\) 的唯一性。

例如,如果与第一个系统相关的系数矩阵 \(\mathbf{A}\) 通过高斯-乔丹方法完全简化为 \(E_A\)
\[
\mathbf{A} = 
\left(\begin{array}{cccc}
    1 & 2 & 2 & 3 \\
    2 & 4 & 1 & 3 \\
    3 & 6 & 1 & 4 \\
\end{array}\right)\to
\left(\begin{array}{cccc}
    1 & 2 & 0 & 1 \\
    0 & 0 & 1 & 1 \\
    0 & 0 & 0 & 0 \\
\end{array}\right)= E_A
\]
然后我们得到以下简化系统:
\[
\begin{aligned}
    x_1 + 2x_2 + x_4 &= 0, \\
    x_3 + x_4 &= 0.
\end{aligned}
\]
用 \(x_2\) 和 \(x_4\) 来求解基本变量 \(x_1\) 和 \(x_3\),得到的结果与当时的推导给出的结果完全相同,因此也得到了与后续的通解完全相同的结果。

由于避免了回代过程,你可能会发现使用高斯-约当法将 \(\mathbf{A}\) 完全简化为 \(\mathbf{E_A}\) 更为方便,
然后直接从 \(\mathbf{E_A}\) 中的元素构建通解。这种方法通常会在示例和练习中采用。

如前所述,所有齐次系统都是一致的,因为由全零组成的平凡解始终是唯一解。一个自然而然的问题是:“平凡解何时是唯一解?”
换句话说,我们想知道齐次系统何时具有唯一解。上述的形式使答案显而易见。只要至少有一个自由变量,那么从上述可以清楚地看出,
将存在无数个解。因此,当且仅当没有自由变量时,平凡解才是唯一解。
由于自由变量的数量为 \(n - r\),其中 \(r\) = rank(\(\mathbf{A}\)),因此前面的陈述可以重新表述为:当且仅当 rank(\(\mathbf{A}\)) = \(n\),齐次系统才具有唯一解——平凡解。

% 红色文字
\textcolor{red}{例子2.4.1}
% 红色水平线(宽度为文本宽度,厚度0.4pt)
\color{red}\rule{\textwidth}{0.4pt}\color{black}

齐次系统
\[
\begin{aligned} 
    x_1 + 2x_2 + 2x_3 = 0, \\
    2x_1 + 5x_2 + 7x_3 = 0, \\
    3x_1 + 6x_2 + 8x_3 = 0.
\end{aligned}
\]
只有平凡解因为
\[
\mathbf{A} = 
\left(\begin{array}{ccc}
    1 & 2 & 2 \\
    2 & 5 & 7 \\
    3 & 6 & 8 \\
\end{array}\right)\to
\left(\begin{array}{cccc}
    1 & 2 & 2 \\
    0 & 1 & 3 \\
    0 & 0 & 2 \\
\end{array}\right) = \mathbf{E}
\]
表明 rank(\(\mathbf{A}\)) = \(n\) = 3。事实上,从 \(\mathbf{E}\) 中也可以看出,在系统 \([\mathbf{E|0}]\) 中应用反向替换只会产生平凡解。

% 红色文字
\textcolor{red}{例子2.4.2}
% 红色水平线(宽度为文本宽度,厚度0.4pt)
\color{red}\rule{\textwidth}{0.4pt}\color{black}

\textbf{问题}: 解释为什么下列齐次系统有无穷多个解,并给出通解:
\[
\begin{aligned} 
    x_1 + 2x_2 + 2x_3 = 0, \\
    2x_1 + 5x_2 + 7x_3 = 0, \\
    3x_1 + 6x_2 + 6x_3 = 0.
\end{aligned}
\]
\textbf{解答}: 
\[
\mathbf{A} = 
\left(\begin{array}{ccc}
    1 & 2 & 2 \\
    2 & 5 & 7 \\
    3 & 6 & 6 \\
\end{array}\right)\to
\left(\begin{array}{cccc}
    1 & 2 & 2 \\
    0 & 1 & 3 \\
    0 & 0 & 0 \\
\end{array}\right) = \mathbf{E}
\]
表明 rank(\(\mathbf{A}\)) = 2 < \(n\) = 3。由于基本列位于位置1 和 2,\(x_1\) 和 \(x_2\) 是基本变量,而 \(x_3\) 是自由变量。使用
\([E|0]\) 的反向代入,用自由变量来求解基本变量,可得 \(x_2 = -3x_3\) 和 \(x_1 = -2x_2 - 2x_3 = 4x_3\),因此通解为
\[
\left(\begin{array}{c}
    x_1 \\
    x_2 \\
    x_3 \\
\end{array}\right) = x_3
\left(\begin{array}{c}
    4 \\
    -3 \\
    1 \\
\end{array}\right), \text{其中} x_3 \text{是自由变量}.
\]
也就是说,每个解都是一个特定解 \(\mathbf{h}_1 = \left(\begin{array}{c} 4 \\-3 \\1\end{array}\right)\) 的倍数

\begin{bluebox}{总结}
设 \(\mathbf{A}_{m \times n}\) 为 \(m\) 个齐次线性方程组的系数矩阵,其中 \(n\) 个未知数,并设 rank(\(\mathbf{A}\)) = \(r\)。
\begin{itemize}
    \item 与基本列的位置(即枢轴位置)相对应的未知数称为\textbf{基本变量},与非基本列的位置相对应的未知数称为\textbf{自由变量}。
    \item 恰好有 \(r\) 个基本变量和 \(n-r\) 个自由变量。
    \item 为了描述所有解,使用高斯消元法将 \(A\) 化为行阶梯形式,然后使用回代法求解,以自由变量的形式表示基本变量。这样就得到了通解,其形式为
    \[
    \mathrm{x} = x_{f_1}\mathbf{h}_1 + x_{f_2}\mathbf{h}_2 + ... + x_{f_{n-r}}\mathbf{h}_{n-r},
    \]
    其中 \(x_{f_1}, x_{f_2}, ..., x_{f_{n-r}}\) 为自由变量,\(\mathbf{h}_1, \mathbf{h}_2, ..., \mathbf{h}_{n-r}\) 为 \(n \times 1\) 列,表示齐次方程组的特解。\(\mathbf{h}_i\) 与回代过程中使用的
    行阶梯形式无关。由于自由变量 \(x_{f_i}\) 涵盖所有可能值,因此通解会生成所有可能的解。
    \item 一个齐次系统拥有唯一解(平凡解),当且仅当 rank(\(\mathbf{A}\)) = \(n\) ——即当且仅当不存在自由变量。
\end{itemize}
\end{bluebox}


% 红色文字
\textcolor{red}{练习 2.4}
% 红色水平线(宽度为文本宽度,厚度0.4pt)
\color{red}\rule{\textwidth}{0.4pt}\color{black}

\begin{enumerate}[leftmargin=*, label=\bfseries 2.4.\arabic*]

\item Determine the general solution for each of the following homogeneous systems.

\begin{enumerate}[label=(\alph*)]
    \item \(\begin{array}{l}
    x_1 + 2x_2 + x_3 + 2x_4 = 0, \\
    2x_1 + 4x_2 + x_3 + 3x_4 = 0, \\
    3x_1 + 6x_2 + x_3 + 4x_4 = 0.
    \end{array}\)
    
    \item \(\begin{array}{l}
    2x + y + z = 0, \\
    4x + 2y + z = 0, \\
    6x + 3y + z = 0, \\
    8x + 4y + z = 0.
    \end{array}\)
    
    \item \(\begin{array}{l}
    x_1 + x_2 + 2x_3 = 0, \\
    3x_1 + 3x_3 + 3x_4 = 0, \\
    2x_1 + x_2 + 3x_3 + x_4 = 0. \\
    x_1 + 2x_2 + 3x_3 - x_4 = 0,
    \end{array}\)

    \item \(\begin{array}{l}
    2x + y + z = 0, \\
    4x + 2y + z = 0, \\
    6x + 3y + z = 0, \\
    8x + 5y + z = 0.
    \end{array}\)
\end{enumerate}

\item Among all solutions that satisfy the homogeneous system
\[
\begin{aligned}
x + 2y + z &= 0, \\
2x + 4y + z &= 0, \\
x + 2y - z &=0.
\end{aligned}
\]
determine those that also satisfy the nonlinear constraint \(y - xy = 2z\).

\item Consider a homogeneous system whose coefficient matrix is
\[
\mathbf{A} = \left(\begin{array}{ccccc}
    1 & 2 & 1 & 3 & 1 \\
    2 & 4 & -1 & 3 & 8 \\
    1 & 2 & 3 & 5 & 7 \\
    2 & 4 & 2 & 6 & 2 \\
    3 & 6 & 1 & 7 & -3
\end{array}\right).
\]
First transform \(\mathbf{A}\) to an unreduced row echelon form to determine the
general solution of the associated homogeneous system. Then reduce \(\mathbf{A}\)
to \(\mathbf{E_A}\), and show that the same general solution is produced.

\item If \(\mathbf{A}\) is the coefficient matrix for a homogeneous system consisting of four equations in eight unknowns and if there are five free variables, what is rank(\(\mathbf{A}\))?

\item Suppose that \(\mathbf{A}\) is the coefficient matrix for a homogeneous system of four equations in six unknowns and suppose that \(\mathbf{A}\) has at least one nonzero row.
\begin{enumerate}[label=(\alph*)]
    \item Determine the fewest number of free variables that are possible.
    \item Determine the maximum number of free variables that are possible.
\end{enumerate}

\item Explain why a homogeneous system of \(m\) equations in \(n\) unknowns where \(m < n\) must always possess an infinite number of solutions.

\item Construct a homogeneous system of three equations in four unknowns that has
\[
x_2\left(\begin{array}{c}
    -2 \\
    1 \\
    0 \\
    0
\end{array}\right)+
x_4\left(\begin{array}{c}
    -1 \\
    0 \\
    -1 \\
    1
\end{array}\right)
\]
as its general solution.

\item If \(\mathbf{c}_1\) and \(\mathbf{c}_2\) are columns that represent two particular solutions of the same homogeneous system, explain why the sum \(\mathbf{c}_1 + \mathbf{c}_2\) must also represent a solution of this system.

\end{enumerate}
