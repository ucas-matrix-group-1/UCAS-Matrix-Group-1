\documentclass[12pt]{article}
\usepackage{ctex}
\usepackage{amsmath, amssymb, amsfonts}
\usepackage{geometry}
\geometry{a4paper, margin=1in}

\title{矩阵分析课程学习报告\\
\large ——四元数与离散傅里叶变换}
\author{蒋  康\ (202518014628074) \\
	何鹏程\ (2025E8017483006) \\
	廖思清\ (202528013229093) \\
	郑钦元\ (202518020629020) \\
	崔晨宇\ (202528013229038) \\
	秦超洋\ (202528013229031)
    }
\date{}

\begin{document}
\maketitle


\section{欧拉角}
\subsection{欧拉角的旋转表示}
欧拉角通过三个角度 $(\phi,\theta,\psi)$ 描述三维旋转。以 $Z\!-\!Y\!-\!Z$ 顺序为例,旋转矩阵表示为
\[
R = R_z(\phi) R_y(\theta) R_z(\psi),
\]
其中
\[
R_z(\alpha)=
\begin{pmatrix}
\cos\alpha & -\sin\alpha & 0\\
\sin\alpha & \cos\alpha & 0\\
0&0&1
\end{pmatrix},\quad
R_y(\beta)=
\begin{pmatrix}
\cos\beta & 0 & \sin\beta\\
0&1&0\\
-\sin\beta & 0 & \cos\beta
\end{pmatrix}.
\]
该表示方式直观,参数数量少,但并非全局有效。

\subsection{万向节死锁问题}
当中间旋转角 $\theta=\frac{\pi}{2}$ 时,上述旋转矩阵中两个旋转轴发生重合,使得系统失去一个自由度。这一现象称为万向节死锁。

从数学角度看,此时欧拉角参数化在该点附近不再是单射,不同的参数组合可能对应同一个旋转矩阵,导致表示不唯一。

\subsection{例子}
当 $\theta=\frac{\pi}{2}$ 时,
\[
(\phi,\theta,\psi)=(0,\tfrac{\pi}{2},\alpha),
\quad
(\phi',\theta',\psi')=(\alpha,\tfrac{\pi}{2},0)
\]
对应的旋转矩阵相同。这说明欧拉角在特定位置存在退化现象,在数值计算和插值过程中容易引发不连续问题。

\section{四元数}
\subsection{四元数的基本定义}
四元数定义为
\[
q = a + b\mathbf{i} + c\mathbf{j} + d\mathbf{k},
\quad a,b,c,d\in\mathbb{R},
\]
其中
\[
\mathbf{i}^2=\mathbf{j}^2=\mathbf{k}^2=\mathbf{ijk}=-1.
\]
在旋转问题中,通常使用模长为 $1$ 的四元数。

\subsection{单位四元数表示旋转}
设旋转轴为单位向量 $\mathbf{u}=(u_x,u_y,u_z)$,旋转角为 $\theta$,则对应的单位四元数为
\[
q = \cos\frac{\theta}{2}
+ \sin\frac{\theta}{2}(u_x\mathbf{i}+u_y\mathbf{j}+u_z\mathbf{k}).
\]
三维向量 $\mathbf{v}$ 可表示为纯虚四元数,通过
\[
\mathbf{v}' = q\,\mathbf{v}\,q^{-1}
\]
实现旋转。

\subsection{例子}
考虑绕 $z$ 轴旋转 $90^\circ$,即
\[
\theta=\frac{\pi}{2},\quad \mathbf{u}=(0,0,1),
\]
对应的单位四元数为
\[
q=\cos\frac{\pi}{4}+\sin\frac{\pi}{4}\mathbf{k}
=\frac{\sqrt{2}}{2}+\frac{\sqrt{2}}{2}\mathbf{k}.
\]
该表示在旋转角度变化时保持连续性,不会出现欧拉角中的奇异情况。

\subsection{四元数相对于欧拉角的优势}
与欧拉角相比,四元数在旋转表示上具有以下优势:
\begin{itemize}
  \item 四元数表示在整个旋转空间内连续,不存在万向节死锁;
  \item 旋转的组合对应四元数乘法,形式统一且计算稳定;
  \item 参数冗余较少,且不存在欧拉角中的角度耦合问题;
  \item 适合用于旋转插值和连续姿态变化的描述。
\end{itemize}



\section{连续傅里叶变换}
\subsection{定义}
连续傅里叶变换定义为
\[
\hat{f}(\omega)=\int_{-\infty}^{\infty} f(t)e^{-i\omega t}\,dt,
\]
其作用是将函数从时域表示转换为频域表示,可视为一种线性变换。

\subsection{示例}
对高斯函数
\[
f(t)=e^{-t^2},
\]
其傅里叶变换为
\[
\hat{f}(\omega)=\sqrt{\pi}e^{-\omega^2/4},
\]

\section{离散傅里叶变换及其矩阵形式}
\subsection{离散傅里叶变换}
在实际计算中,函数只能在有限个采样点上取值。设离散序列为
\[
x_0,x_1,\dots,x_{N-1},
\]
其离散傅里叶变换定义为
\[
X_k=\sum_{n=0}^{N-1} x_n e^{-2\pi i kn/N},
\quad k=0,1,\dots,N-1.
\]

\subsection{傅里叶矩阵}
定义 $N\times N$ 傅里叶矩阵
\[
F_N =
\begin{pmatrix}
1 & 1 & 1 & \cdots & 1\\
1 & \omega & \omega^2 & \cdots & \omega^{N-1}\\
1 & \omega^2 & \omega^4 & \cdots & \omega^{2(N-1)}\\
\vdots & \vdots & \vdots & \ddots & \vdots\\
1 & \omega^{N-1} & \omega^{2(N-1)} & \cdots & \omega^{(N-1)^2}
\end{pmatrix},
\quad
\omega=e^{-2\pi i/N}.
\]
则离散傅里叶变换可写为矩阵形式
\[
\mathbf{X}=F_N\mathbf{x}.
\]
在适当归一化后,$F_N$ 为酉矩阵。

IDFT(逆变换)可表示为:
\[
x = \frac{1}{N} F_N^* X
\]
其中 $F_N^*$ 是 $F_N$ 的共轭转置。

\subsection{傅里叶矩阵的性质}
\begin{enumerate}
    \item \textbf{对称性}:$F_N$ 是对称矩阵(但非Hermitian)
    \item \textbf{正交性}:$\frac{1}{\sqrt{N}} F_N$ 是酉矩阵,即:
    \[
    F_N F_N^* = F_N^* F_N = N I_N
    \]
    其中 $I_N$ 是 $N \times N$ 单位矩阵。
    \item \textbf{可逆性}:$F_N^{-1} = \frac{1}{N} F_N^*$
    \item \textbf{循环结构}:矩阵的每一行(或列)都是上一行(或列)的循环移位,乘上 $W_N$ 的幂次。
\end{enumerate}

\subsection{例子}
取 $N=4$,则
\[
\omega=e^{-2\pi i/4}=e^{-i\pi/2}=-i,
\quad
\omega^2=-1,\ \omega^3=i,\ \omega^4=1.
\]
傅里叶矩阵 $F_4$ 具体为
\[
F_4=
\begin{pmatrix}
1 & 1 & 1 & 1\\
1 & \omega & \omega^2 & \omega^3\\
1 & \omega^2 & \omega^4 & \omega^6\\
1 & \omega^3 & \omega^6 & \omega^9
\end{pmatrix}
=
\begin{pmatrix}
1 & 1 & 1 & 1\\
1 & -i & -1 & i\\
1 & -1 & 1 & -1\\
1 & i & -1 & -i
\end{pmatrix},
\]
其中用到了 $\omega^4=1,\ \omega^6=\omega^2=-1,\ \omega^9=\omega^{8}\omega=\omega=-i$。

令
\[
\mathbf{x}=(1,0,-1,0)^T,
\quad \mathbf{X}=F_4\mathbf{x}.
\]
则按行相乘得到
\[
X_0=
\begin{pmatrix}1&1&1&1\end{pmatrix}
\begin{pmatrix}1\\0\\-1\\0\end{pmatrix}
=1+0-1+0=0,
\]
\[
X_1=
\begin{pmatrix}1&-i&-1&i\end{pmatrix}
\begin{pmatrix}1\\0\\-1\\0\end{pmatrix}
=1+0+(-1)\cdot(-1)+0=2,
\]
\[
X_2=
\begin{pmatrix}1&-1&1&-1\end{pmatrix}
\begin{pmatrix}1\\0\\-1\\0\end{pmatrix}
=1+0+(-1)\cdot 1+0=0,
\]
\[
X_3=
\begin{pmatrix}1&i&-1&-i\end{pmatrix}
\begin{pmatrix}1\\0\\-1\\0\end{pmatrix}
=1+0+(-1)\cdot(-1)+0=2.
\]
因此
\[
\mathbf{X}=(0,2,0,2)^T.
\]

\subsection{频率分量的物理意义}
\begin{itemize}
    \item $X[0]$:直流分量,序列的平均值
    \item $X[1]$:基频分量,周期为 $N$
    \item $X[k]$:第 $k$ 次谐波,频率为基频的 $k$ 倍
    \item $X[N-k]$:与 $X[k]$ 共轭,表示负频率成分(由于对称性)
\end{itemize}

\end{document}
