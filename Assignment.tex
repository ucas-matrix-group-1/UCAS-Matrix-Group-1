\documentclass[12pt]{article}
\usepackage[UTF8]{ctex}
\usepackage{amsmath,amssymb,amsthm}
\usepackage{geometry}
\usepackage{booktabs}
\usepackage{enumitem}
\usepackage{graphicx}
\usepackage{hyperref}
\geometry{a4paper, margin=1in}

\title{学习报告:四元数与离散傅里叶变换}
\author{
	蒋  康\ (202518014628074) \\
	何鹏程\ (2025E8017483006) \\
	廖思清\ (202528013229093) \\
	郑钦元\ (202518020629020) \\
	崔晨宇\ (202528013229038) \\
	秦超洋\ (202528013229031)
}
\date{}

\begin{document}
	
	\maketitle
	
	\section*{一、四元数(Quaternions)}
	\subsection*{1.1 基本定义与表示}
	四元数是复数在四维空间中的推广,由一个实部和三个虚部构成:
	\[
	q = a + bi + cj + dk \quad (a, b, c, d \in \mathbb{R})
	\]
	其中:
	\begin{itemize}
		\item $a$ 为实部(scalar part)
		\item $bi + cj + dk$ 为虚部(vector part)
		\item $i, j, k$ 为四元数单位元,满足以下乘法规则:
		\[
		i^2 = j^2 = k^2 = -1, \quad ij = k, \quad jk = i, \quad ki = j
		\]
		\[
		ji = -k, \quad kj = -i, \quad ik = -j, \quad ijk = -1
		\]
	\end{itemize}
	
	\subsection*{1.2 核心运算}
	\subsubsection*{1.2.1 四元数乘法}
	给定两个四元数 $q_1 = a_1 + b_1i + c_1j + d_1k$,$q_2 = a_2 + b_2i + c_2j + d_2k$:
	\[
	q_1q_2 = (a_1a_2 - b_1b_2 - c_1c_2 - d_1d_2) + (a_1b_2 + b_1a_2 - c_1d_2 - d_1c_2)i + (a_1c_2 - b_1d_2 + c_1a_2 + d_1b_2)j + (a_1d_2 + b_1c_2 - c_1b_2 + d_1a_2)k
	\]
	这个公式可通过乘法分配律和单位元规则推导得出。
	
	\subsubsection*{1.2.2 共轭与模长}
	\begin{itemize}
		\item \textbf{共轭四元数}:
		\[
		q^* = a - bi - cj - dk
		\]
		\item \textbf{模长}:
		\[
		\|q\| = \sqrt{q q^*} = \sqrt{a^2 + b^2 + c^2 + d^2}
		\]
		\item \textbf{单位四元数}:$\|q\| = 1$,用于表示旋转。
	\end{itemize}
	
	\subsection*{1.3 几何意义:三维旋转表示}
	\subsubsection*{1.3.1 旋转公式}
	对于一个三维向量 $v = (x, y, z)$,可视为纯四元数 $v = 0 + xi + yj + zk$。  
	绕单位轴 $u = (u_x, u_y, u_z)$ 旋转角度 $\theta$ 的操作为:
	\[
	v' = qvq^*
	\]
	其中 $q = \cos \frac{\theta}{2} + (u_x i + u_y j + u_z k) \sin \frac{\theta}{2}$ 为单位四元数。
	
	\subsubsection*{1.3.2 理解思考:为何用半角?}
	这是四元数表示中最巧妙也最令人困惑的地方。推导过程显示:
	\begin{enumerate}
		\item 旋转操作需要两次乘法($qvq^*$)
		\item 每个乘法都引入一半的旋转效果
		\item 因此旋转角度需要减半来补偿
	\end{enumerate}
	实际验证:设 $v$ 为与旋转轴垂直的向量,计算 $v'$ 会得到旋转 $\theta$ 后的结果。
	
	\subsection*{1.4 优势与应用}
	\subsubsection*{1.4.1 相比于欧拉角}
	\begin{itemize}
		\item \textbf{无方向节死锁}:四元数用四个参数表示三维旋转,不会出现奇异点
		\item \textbf{插值自然}:球面线性插值(SLERP)在四元数单位球面上沿大圆路径移动,旋转平滑
		\item \textbf{计算高效}:旋转组合只需四元数乘法,比矩阵乘法计算量小
	\end{itemize}
	
	\subsubsection*{1.4.2 相比于旋转矩阵}
	\begin{itemize}
		\item \textbf{存储更省}:四元数(4个浮点数) vs 旋转矩阵(9个浮点数)
		\item \textbf{数值稳定}:容易保持单位长度,矩阵可能因累积误差失去正交性
	\end{itemize}
	
	\subsubsection*{1.4.3 主要应用领域}
	\begin{itemize}
		\item 计算机图形学:相机控制、角色动画
		\item 机器人学:姿态估计、路径规划
		\item 航空航天:飞行器姿态控制
		\item VR/AR:头部追踪与方向计算
	\end{itemize}
	
	\section*{二、离散傅里叶变换(Discrete Fourier Transform, DFT)}
	\subsection*{2.1 基本定义与公式}
	对于长度为 $N$ 的离散序列 $x[n]$($n = 0, 1, ..., N - 1$),其DFT定义为:
	\[
	X[k] = \sum_{n=0}^{N-1} x[n] \cdot e^{-j\frac{2\pi}{N}kn}, \quad k = 0, 1, ..., N - 1
	\]
	逆变换(IDFT)为:
	\[
	x[n] = \frac{1}{N} \sum_{k=0}^{N-1} X[k] \cdot e^{j\frac{2\pi}{N}kn}, \quad n = 0, 1, ..., N - 1
	\]
	
	\subsection*{2.2 傅里叶矩阵:理解DFT的线性代数视角}
	\subsubsection*{2.2.1 定义傅里叶矩阵}
	令 $W_N = e^{-j\frac{2\pi}{N}}$(旋转因子,第 $N$ 次单位根),则 $N \times N$ 的傅里叶矩阵 $F_N$ 定义为:
	\[
	F_N = 
	\begin{bmatrix}
		W_N^{0,0} & W_N^{0,1} & \cdots & W_N^{0(N-1)} \\
		W_N^{1,0} & W_N^{1,1} & \cdots & W_N^{1(N-1)} \\
		\vdots & \vdots & \ddots & \vdots \\
		W_N^{(N-1),0} & W_N^{(N-1),1} & \cdots & W_N^{(N-1)(N-1)}
	\end{bmatrix}
	\]
	即 $(F_N)_{k,n} = W_N^{kn} = e^{-j\frac{2\pi}{N}kn}$。
	
	\subsubsection*{2.2.2 DFT的矩阵表示}
	将信号向量化:$x = [x[0], x[1], ..., x[N-1]]^T$ \\
	将频谱向量化:$X = [X[0], X[1], ..., X[N-1]]^T$ \\
	则DFT可表示为:
	\[
	X = F_N x
	\]
	IDFT可表示为:
	\[
	x = \frac{1}{N} F_N^* X
	\]
	其中 $F_N^*$ 是 $F_N$ 的共轭转置。
	
	\subsubsection*{2.2.3 傅里叶矩阵的性质}
	\begin{enumerate}
		\item \textbf{对称性}:$F_N$ 是对称矩阵(但非Hermitian)
		\item \textbf{正交性}:$\frac{1}{\sqrt{N}} F_N$ 是酉矩阵,即:
		\[
		F_N F_N^* = F_N^* F_N = N I_N
		\]
		其中 $I_N$ 是 $N \times N$ 单位矩阵。
		\item \textbf{可逆性}:$F_N^{-1} = \frac{1}{N} F_N^*$
		\item \textbf{循环结构}:矩阵的每一行(或列)都是上一行(或列)的循环移位,乘上 $W_N$ 的幂次。
	\end{enumerate}
	
	\subsection*{2.3 DFT的深入理解}
	\subsubsection*{2.3.1 频率分量的物理意义}
	\begin{itemize}
		\item $X[0]$:直流分量,序列的平均值
		\item $X[1]$:基频分量,周期为 $N$
		\item $X[k]$:第 $k$ 次谐波,频率为基频的 $k$ 倍
		\item $X[N-k]$:与 $X[k]$ 共轭,表示负频率成分(由于对称性)
	\end{itemize}
	
	\subsubsection*{2.3.2 周期性假设}
	DFT隐含假设信号以 $N$ 为周期延拓。这意味着:
	\begin{itemize}
		\item 频域离散化对应时域周期化
		\item 时域离散化对应频域周期化(周期为采样率 $f_s$)
		\item 实际应用中需注意频谱泄漏和栅栏效应
	\end{itemize}
	
	\subsection*{2.4 快速傅里叶变换(FFT)原理}
	\subsubsection*{2.4.1 分治思想}
	FFT基于Cooley-Tukey算法,核心是将DFT分解为更小的DFT:
	\[
	X[k] = \sum_{n=0}^{N-1} x[n] W_N^{kn} = \sum_{m=0}^{N/2-1} x[2m] W_N^{k(2m)} + \sum_{m=0}^{N/2-1} x[2m+1] W_N^{k(2m+1)}
	\]
	令 $W_N^2 = W_{N/2}$,则:
	\[
	X[k] = \sum_{m=0}^{N/2-1} x[2m] W_{N/2}^{km} + W_N^k \sum_{m=0}^{N/2-1} x[2m+1] W_{N/2}^{km}
	\]
	
	\subsubsection*{2.4.2 矩阵分解视角}
	FFT相当于对傅里叶矩阵进行因式分解:
	\[
	F_N = P_N \begin{bmatrix}
		F_{N/2} & D_{N/2} F_{N/2} \\
		F_{N/2} & -D_{N/2} F_{N/2}
	\end{bmatrix}
	\]
	其中:
	\begin{itemize}
		\item $P_N$:置换矩阵(重排偶数和奇数索引)
		\item $D_{N/2}$:对角矩阵,元素为 $W_N^0, W_N^1, ..., W_N^{N/2-1}$
	\end{itemize}
	
	\subsubsection*{2.4.3 算法复杂度}
	\begin{itemize}
		\item 直接DFT:$O(N^2)$ 次复数乘法
		\item FFT:$O(N \log_2 N)$ 次复数乘法
		\item 当 $N = 1024$ 时,加速比约为 $1024 / \log_2 1024 \approx 100$
	\end{itemize}
	
	\subsection*{2.5 DFT的应用领域}
	\subsubsection*{2.5.1 信号处理}
	\begin{itemize}
		\item \textbf{频谱分析}:语音识别、振动分析
		\item \textbf{滤波}:频域乘法等效于时域卷积
		\item \textbf{相关分析}:计算信号相似度
	\end{itemize}
	
	\subsubsection*{2.5.2 图像处理}
	\begin{itemize}
		\item \textbf{二维DFT}:图像频域分析
		\item \textbf{图像压缩}:JPEG使用DCT(DFT的实部变换)
		\item \textbf{图像滤波}:高通/低通/带通滤波
	\end{itemize}
	
	\subsubsection*{2.5.3 通信系统}
	\begin{itemize}
		\item OFDM:正交频分复用,利用DFT/IDFT实现多载波调制
		\item \textbf{信道估计}:通过导频信号估计频率响应
	\end{itemize}
	
	\section*{三、概念对比与综合思考}
	\subsection*{3.1 数学结构的相似性}
	\begin{table}[htbp]
		\centering
		\begin{tabular}{@{}lll@{}}
			\toprule
			特性 & 四元数 & DFT(傅里叶矩阵) \\
			\midrule
			维度扩展 & 复数→四元数(四维) & 实数/复数→频域表示 \\
			乘法规则 & 非交换代数 & 矩阵乘法(可交换,因傅里叶矩阵可对角化) \\
			正交性 & 单位四元数构成三维球面 & 傅里叶矩阵列向量正交 \\
			变换操作 & 共轭乘法表示旋转 & 矩阵乘法表示时频变换 \\
			\bottomrule
		\end{tabular}
	\end{table}
	
	\subsection*{3.2 理解难点的共性分析}
	\subsubsection*{3.2.1 抽象维度}
	\begin{itemize}
		\item 四元数:用四维代数对象表示三维旋转,需建立几何直观
		\item DFT:时域到频域的变换,需理解复数频率的物理意义
	\end{itemize}
	
	\subsubsection*{3.2.2 符号与记法}
	\begin{itemize}
		\item 四元数:三个虚单位 $i, j, k$ 及其非交换乘法
		\item DFT:旋转因子 $W_N = e^{-j2\pi/N}$ 的幂次运算
	\end{itemize}
	
	\subsubsection*{3.2.3 隐含假设}
	\begin{itemize}
		\item 四元数旋转:默认向量表示为纯四元数
		\item DFT:默认信号周期延拓,采样定理前提
	\end{itemize}
	
	\subsection*{3.3 工程应用的互补性}
	在实际系统中,两者可能协同工作:
	\begin{enumerate}
		\item \textbf{运动捕捉系统}:传感器数据(加速度、角速度)通过DFT分析频率特征,姿态通过四元数融合
		\item \textbf{机器人控制}:路径规划中的轨迹用DFT分析,末端执行器姿态用四元数插值
		\item \textbf{计算机视觉}:图像处理用DFT/FFT,三维重建中的旋转用四元数表示
	\end{enumerate}
	
	\subsection*{3.4 学习心得与启示}
	\subsubsection*{3.4.1 四元数学习要点}
	\begin{enumerate}
		\item 从复数旋转(二维)类比到四元数旋转(三维)
		\item 用具体例子验证旋转公式,建立几何直观
		\item 理解SLERP插值的几何意义:四维单位球面上的大圆弧
	\end{enumerate}
	
	\subsubsection*{3.4.2 DFT学习要点}
	\begin{enumerate}
		\item 从傅里叶级数→连续FT→离散FT的演变过程
		\item 通过傅里叶矩阵理解DFT的线性代数本质
		\item 手动推导小N值(如N=4)的FFT模型,理解分治策略
	\end{enumerate}
	
	\subsubsection*{3.4.3 计算实现建议}
	\begin{itemize}
		\item 四元数:实现四元数,重载乘法运算符,验证旋转性质
		\item DFT/FFT:先实现直接DFT(理解原理),再实现递归FFT,最后迭代优化
	\end{itemize}
	
	\section*{四、总结}
	四元数和离散傅里叶变换虽然属于不同数学分支,但都体现了\textbf{用巧妙数学工具解决实际问题的思想}:
	\begin{enumerate}
		\item \textbf{四元数}通过扩展复数系统,以紧凑、稳定的方式处理三维旋转问题,避免了欧拉角的万向节死锁和旋转矩阵的冗余。
		\item \textbf{DFT}通过傅里叶矩阵将信号分解为频率分量,FFT算法极大提升了计算效率,成为现代数字信号处理的基石。
	\end{enumerate}
	
	理解这两个概念的关键在于:
	\begin{itemize}
		\item 把握从简单到复杂的推广过程(复数→四元数,连续FT→离散FT)
		\item 建立几何物理直观,而不仅停留在公式层面
		\item 通过编程实现加深理解,观察参数变化的影响
	\end{itemize}
	
	这两项技术将继续在计算机图形学、机器人、通信等领域发挥核心作用,是现代工程师和研究人员的重要数学工具。
	
	\section*{参考文献}
	\begin{enumerate}
		\item Kuipers, J. B. (1999). \textit{Quaternions and Rotation Sequences}
		\item Oppenheim, A. V., \& Schafer, R. W. (2010). \textit{Discrete-Time Signal Processing}
		\item Cooley, J. W., \& Tukey, J. W. (1965). \textit{An algorithm for the machine calculation of complex Fourier series}
	\end{enumerate}
	
\end{document}