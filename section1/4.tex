\subsection{两点边值问题}

之前曾提到,在实际应用中产生的线性系统规模往往相当大。本节的目的是理解为什么这种情况经常发生,以及为什么实际应用中产生的线性系统常常具有特殊的结构。

给定区间 \([a, b]\) 和两个数 \(\alpha\) 和 \(\beta\),考虑寻找函数 \(y(t)\) 满足微分方程的一般问题:
\[ u(t)y''(t) + v(t)y'(t) + w(t)y(t) = f(t), \quad \text{其中 } y(a) = \alpha \text{ 且 } y(b) = \beta. \tag{1.4.1} \]

函数 \(u\)、\(v\)、\(w\) 和 \(f\) 在 \([a, b]\) 上假定为已知函数。由于未知函数 \(y(t)\) 在边界点 \(a\) 和 \(b\) 处被指定,问题 (1.4.1) 被称为\textbf{两点边值问题}。这类问题在自然界中大量存在,并且通常很难处理,因为通常不可能用初等函数表示 \(y(t)\)。数值方法通常用于在 \([a, b]\) 内的离散点处近似 \(y(t)\)。通过将区间 \([a, b]\) 细分为 \(n+1\) 个长度均为 \(h = (b - a)/(n + 1)\) 的相等子区间来产生近似值。

在内部节点(网格点)\(t_i = a + ih\)处的导数近似是通过使用泰勒级数展开 \(y(t) = \sum_{k=0}^{\infty} y^{(k)}(t_i)(t - t_i)^k / k!\) 来实现的,即
\[
\begin{aligned}
y(t_i + h) &= y(t_i) + y'(t_i)h + \frac{y''(t_i)h^2}{2!} + \frac{y'''(t_i)h^3}{3!} + \cdots, \\
y(t_i - h) &= y(t_i) - y'(t_i)h + \frac{y''(t_i)h^2}{2!} - \frac{y'''(t_i)h^3}{3!} + \cdots,
\end{aligned} \tag{1.4.2}
\]
然后对这些表达式进行相减和相加,得到
\[
y'(t_i) = \frac{y(t_i + h) - y(t_i - h)}{2h} + O(h^3)
\]
和
\[
y''(t_i) = \frac{y(t_i - h) - 2y(t_i) + y(t_i + h)}{h^2} + O(h^4),
\]
其中 \(O(h^p)\) 表示\footnote{形式上,如果 \(f(h)/h^p\) 在 \(h \to 0\) 时保持有界,但 \(f(h)/h^q\) 在 \(q > p\) 时变为无界,则函数 \(f(h)\) 是 \(O(h^p)\)。这意味着 \(f\) 趋于零的速度与 \(h^p\) 趋于零的速度一样快。}包含 \(h\) 的 \(p\) 次及更高次幂的项。

由此得到的近似
\[ y'(t_i) \approx \frac{y(t_i + h) - y(t_i - h)}{2h} \text{ 和 } y''(t_i) \approx \frac{y(t_i - h) - 2y(t_i) + y(t_i + h)}{h^2} \tag{1.4.3} \]
被称为\textbf{中心差分近似},它们优于精度较低的单侧近似,例如
\[ y'(t_i) \approx \frac{y(t_i + h) - y(t_i)}{h} \text{ 或 } y'(t_i) \approx \frac{y(t_i) - y(t_i - h)}{h}. \]

值 \( h = (b - a)/(n + 1) \) 被称为\textbf{步长}。更小的步长产生更好的导数近似,因此获得准确的解通常需要小步长和大量网格点。通过在每个网格点处计算中心差分近似并将结果代入原始微分方程 (1.4.1),产生一个包含 \( n \) 个未知数的 \( n \) 个线性方程的系统,其中未知数是值 \( y(t_i) \)。一个简单的例子可以用来说明这一点。

% 红色文字
\textcolor{red}{例子1.4.1}
% 红色水平线(宽度为文本宽度,厚度0.4pt)
\color{red}\rule{\textwidth}{0.4pt}\color{black}

问题:假设 \( f(t) \) 是一个已知函数,考虑两点边值问题
\[ y''(t) = f(t) \text{ 在 } [0, 1] \text{ 上,且 } y(0) = y(1) = 0. \]

目标是在 \([0, 1]\) 内部的 \( n \) 个等间距网格点 \( t_i \) 处近似 \( y \) 的值。因此步长为 \( h = 1/(n + 1) \)。为方便起见,令 \( y_i = y(t_i) \) 和 \( f_i = f(t_i) \)。使用近似
\[ \frac{y_{i-1} - 2y_i + y_{i+1}}{h^2} \approx y''(t_i) = f_i \]
结合 \( y_0 = 0 \) 和 \( y_{n+1} = 0 \),得到方程组
\[ -y_{i-1} + 2y_i - y_{i+1} \approx -h^2 f_i \quad \text{对于 } i = 1, 2, \ldots, n. \]

(选择符号使2为正,以与后续发展保持一致。)与该系统相关的增广矩阵如下所示:
\[
\left(
\begin{array}{ccccccc|c}
2 & -1 & 0 & \cdots & 0 & 0 & 0 & -h^2 f_1 \\
-1 & 2 & -1 & \cdots & 0 & 0 & 0 & -h^2 f_2 \\
0 & -1 & 2 & \cdots & 0 & 0 & 0 & -h^2 f_3 \\
\vdots & \vdots & \vdots & \ddots & \vdots & \vdots & \vdots & \vdots \\
0 & 0 & 0 & \cdots & 2 & -1 & 0 & -h^2 f_{n-2} \\
0 & 0 & 0 & \cdots & -1 & 2 & -1 & -h^2 f_{n-1} \\
0 & 0 & 0 & \cdots & 0 & -1 & 2 & -h^2 f_n \\
\end{array}
\right)
\]

通过求解该系统,得到未知函数 \( y \) 在网格点 \( t_i \) 处的近似值。更大的 \( n \) 值产生更小的 \( h \) 值,从而产生对 \( y_i \) 的精确值更好的近似。

% 红色水平线(宽度为文本宽度,厚度0.4pt)
\color{red}\rule{\textwidth}{0.4pt}\color{black}

注意上例中系数矩阵中元素的模式。非零元素仅出现在次对角线、主对角线和超对角线上——这样的系统(或矩阵)被称为\textbf{三对角矩阵}。这是具有特征性的,即当有限差分近似应用于一般两点边值问题时,结果是一个三对角系统。

三对角系统特别好处理,因为求解成本低廉。当应用高斯消元时,三角化过程的每一步只需要两次乘法/除法,因为每个主元的下方和右侧最多只有一个非零元素。此外,高斯消元保留了原始三对角系统中存在的所有零元素。这使得回代过程的执行成本低廉,因为每个替代步骤最多只需要两次乘法/除法。练习 3.10.6 包含更多细节。

% 红色文字
\textcolor{red}{练习1.4}
% 红色水平线(宽度为文本宽度,厚度0.4pt)
\color{red}\rule{\textwidth}{0.4pt}\color{black}

\textbf{1.4.1}将区间 \([0, 1]\) 分为五个相等的子区间,并应用有限差分方法以近似两点边值问题
\[ y''(t) = 125t, \quad y(0) = y(1) = 0 \]
在四个内部网格点的解。将你在网格点处的近似值与网格点处的精确解进行比较。\textbf{注意:}仅用四个内部网格点,你不应期望得到非常精确的近似。

\textbf{1.4.2}将 \([0, 1]\) 分为 \(n+1\) 个相等的子区间,并应用有限差分近似方法推导与两点边值问题
\[ y''(t) - y'(t) = f(t), \quad y(0) = y(1) = 0 \]
相关的线性系统。

\textbf{1.4.3}将 \([0, 1]\) 分为五个相等的子区间,并在四个内部网格点处近似以下问题的解:
\[ y''(t) - y'(t) = 125t, \quad y(0) = y(1) = 0. \]
将近似值与网格点处的精确值进行比较。