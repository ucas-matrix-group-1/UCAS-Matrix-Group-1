\subsection{让高斯消元法work}

你已经了解了高斯消元法的基本思想,让我们将其应用到实际问题中。
对于用笔和纸张进行的计算,我们一般是让情况尽可能简单,比如避免分数的出现,
以尽量减少计算失误的可能性。但是,现实世界的大部分问题没有这么“友好”。涉及
线性系统的实际应用我们通常会用到计算机。计算机不会采用分数,而是使用浮点数。
使用浮点数会产生一种可以预测的舍入误差,我们需要先了解这种误差和它对求解线性系统的影响。

计算机用浮点数来逼近实数,表示形式如下:

\begin{bluebox}{浮点数}
    一个$t$位,基数为$\beta$的\textit{\textbf{浮点数}}具有如下形式:
    $$f = \pm .d_{1} d_{2} \cdots d_{t} \times \beta^{\epsilon} \ \text{with} \ d_{1} \neq 0$$
    其中,基数$\beta$,指数$\epsilon$都是整数,且$d_{i}$是介于$0$和$\beta - 1$之间的整数。对于计算机而言,
    $\beta=2$比较常用,在纸笔上进行运算,通常取$\beta=10$。$t$和$\epsilon$的值取决于实际的硬件和软件。
\end{bluebox}

浮点数的表示形式只是我们熟悉的科学计数法的变体,在本文例子中我们一般取$\beta=10$。对于任意给定的$t$,$\beta$和$\epsilon$,显然它们能够表示的
实数集合$F$是有限集合。用浮点数来逼近实数的方法不止一种,因此,我们采用如下舍入约定:

给定一个实数$x$,$fl(x)$表示将$x$舍入到最接近的浮点数。如果$x$正好位于两个浮点数的中间,则取绝对值更大的那个浮点数。
取$t$位精度近似,当$\beta=10$,我们观察$x=.d_1d_2\cdots d_td_{t+1} \times 10^{\epsilon}$的$d_{t+1}$位($d_1\neq0$),
然后可以得到:
$$fl(x) = \begin{cases} .d_1d_2\cdots d_t \times 10^\epsilon & \text{if } d_{t+1} < 5, \\ ([.d_1d_2\cdots d_t] + 10^{-t}) \times 10^\epsilon & \text{if } d_{t+1} \ge 5. \end{cases}$$
举个例子, 在取$2$位精度近似,以10位基数的情况下,
$$fl(3/80) = fl(.0375) = fl(.375 \times 10^{-1}) = .38 \times 10^{-1} = .038.$$
考虑$\eta = 21/2$ 和$\xi =11/2$, 显然:
$$fl(\eta + \xi) \neq fl(\eta) + fl(\xi) \text{ 且\ } fl(\eta \times \xi) \neq fl(\eta) \times fl(\xi)$$

此外,实数运算中的若干常用法则在浮点运算中并不成立,结合律就是一个显著的例子。这一点,连同其他原因,使得浮点计算的分析工作颇具难度。
这一特性还意味着,你在处理本书中的例题与习题时必须格外谨慎。原因如下:尽管大多数计算器和计算机可被设置为显示不同位数的数字,但它们在显示结果前,所有计算都基于一个固定的内部精度完成,且该精度无法更改。你的计算器的内部精度,通常高于本书例题与习题所要求的精度。

因此,每当你用计算器进行$t$位精度的计算时,都应按以下步骤操作:

\begin{enumerate}
    \item 手动将计算结果舍入到$t$位有效数字。
    \item 在进行下一步计算前,将舍入后的数字重新手动输入计算器。
\end{enumerate}
换言之,不要在计算器或计算机中 “串联” 执行多个运算步骤。

为了更加详细了解如何使用浮点运算执行高斯消元法,让我们比较一下使用精确运算和使用$3$位精度和以$10$为基数的运算来求解以下系统:

$$\begin{cases}
    47x + 28y = 19 \\
    89x + 53y = 36
\end{cases}$$

使用精确计算的高斯消元法时,将第一个方程乘以乘数$m = 89/47$,然后从第二个方程中减去结果,得到:
$$\left(\begin{array}{cc|c}
    47 & 28 & 19 \\
    0 & -1/47 & 1/47
\end{array}\right) $$

回代得到精确解为$x=1$和$y=-1$。
使用$3$位精度浮点数进行高斯消元法时,乘数为:
$$fl(m) = fl(89/47) = .189 \times 10^1 = 1.89$$
由于
$$fl(fl(m) \times fl(47)) = fl(1.89 \times 47) = .888 \times 10^2 = 88.8$$
$$fl(fl(m) \times fl(28)) = fl(1.89 \times 28) = .529 \times 10^2 = 52.9$$
$$fl(fl(m) \times fl(19)) = fl(1.89 \times 19) = .359 \times 10^2 = 35.9$$
$3$位精度的高斯消元法第一步如下:
$$
\left(\begin{array}{cc|c}
    47 & 28 & 19 \\
    fl(89-88.8) & fl(53-52.9) & fl(36-35.9)
\end{array}\right) =
$$
$$
\left(\begin{array}{cc|c}
    47 & 28 & 19 \\
    \circlednum{.2} & .1 & .1
\end{array}\right)
$$
目标是将方程组三角化——在标记的$(2,1)$位置产生一个零,但使用3位精度消元法无法实现这一点。除非将标记的$\circlednum{.2}$替换为$0$,否则无法执行回代。此后,我们约定:\textit{无论实际出现的浮点数是多少,只需在要消去的位置填入$0$}。例如,在上述示例中,甚至无需计算
$$fl(89 - fl(fl(m) \times fl(47))) = fl(89 - 88.8) = 0.2$$
因此,上述示例的3位高斯消元法结果为:

$$
\left(\begin{array}{cc|c} 
47 & 28 & 19 \\
0 & .1 & .1
\end{array}\right)
$$
再应用3位精度回代,得到解为:
$$y = fl(.1/.1) = 1$$
$$x = fl(\frac{19-28}{47})=fl(\frac{-9}{47})=-.191$$
精确解$(1, -1)$与3位精度解$(-0.191, 1)$之间的巨大差异,说明了在尝试用浮点数算术求解线性方程组时可能遇到的一些问题。有时使用更高的精度可能会有所帮助,但这并不总是可行的——所有设备都有自然限制,使得超过一定程度后,扩展精度算术变得不切实际。即使有可能提高精度,也可能无法带来太多好处,因为在许多情况下,精度的提高并不会使累积的舍入误差相应减少。对于任何特定的精度(例如$t$),都不难构造出线性方程组的示例,其计算出的$t$位解与上述3位精度示例中的解一样糟糕。

尽管舍入的影响几乎永远无法消除,但有一些简单的技术可以帮助最小化这些机器引起的误差。

\begin{bluebox}{部分主元法}
在每一步,搜索主元位置及其下方的所有位置,找到绝对值最大的系数。如有必要,执行相应的行交换,将这个最大系数带入主元位置。下面展示了典型情况下的第三步:
$$
\left(\begin{array}{cccccc|c} 
* & * & * & * & * & * & * \\
0 & * & * & * & * & * & * \\
0 & 0 & \circlednum{s} & * & * & * & * \\
0 & 0 & S & * & * & * & * \\
0 & 0 & S & * & * & * & *
\end{array}\right)
$$
搜索第三列中标记为“\textbf{S}”的位置,找到绝对值最大的系数;如有必要,交换行,将该系数带入标记的主元位置。简而言之,策略是通过仅使用行交换,在每一步最大化主元的绝对值。


\end{bluebox}
表面上看,部分主元法为何能产生影响可能并不明显。下面的示例不仅表明部分主元法确实能产生很大的差异,还说明了该策略有效的原因。

\begin{example}
\label{example1.5.1}
容易验证方程组:
$$
\begin{cases} 
-10^{-4}x + y = 1, \\
x + y = 2
\end{cases}
$$
的精确解为:
$$
x = \frac{1}{1.0001}, \quad y = \frac{1.0002}{1.0001}
$$
若使用3位精度但不使用部分主元法,结果为:
$$
\left(\begin{array}{cc|c} 
-10^{-4} & 1 & 1 \\
1 & 1 & 2
\end{array}\right) \xrightarrow{R_2 + 10^4 R_1} \left(\begin{array}{cc|c} 
-10^{-4} & 1 & 1 \\
0 & 10^4 & 10^4
\end{array}\right)
$$
因为:
\begin{equation}
fl(1+10^4)=fl(.10001 \times 10^5) = .100 \times 10^5=10^4 \label{equation1.5.1}
\end{equation}
\begin{equation}
fl(2+10^4)=fl(.10002 \times 10^5) = .100 \times 10^5=10^4 \label{equation1.5.2}    
\end{equation}
回代可得:
$$
x = 0 \text{\ 且\ } y = 1
$$
尽管计算出的$y$的解接近精确解中的$y$,但计算出的$x$的解与精确解中的$x$相差甚远——计算出的$x$的解显然没有达到预期的三位有效数字精度。若使用3位精度且采用部分主元法,结果为:
$$
\left(\begin{array}{cc|c} 
-10^{-4} & 1 & 1 \\
1 & 1 & 2
\end{array}\right) \xrightarrow{} \left(\begin{array}{cc|c} 
1 & 1 & 2 \\
-10^{-4} & 1 & 1
\end{array}\right) \xrightarrow{R_2 + 10^{-4} R_1} \left(\begin{array}{cc|c} 
1 & 1 & 2 \\
0 & 1 & 1
\end{array}\right)
$$
因为:
\begin{equation}
fl(1 + 10^{-4}) = fl(.10001 \times 10^1) = .100 \times 10^1 = 1 \label{equation1.5.3}
\end{equation}
\begin{equation}
fl(1 + 2 \times 10^{-4}) = fl(0.10002 \times 10^1) = .100 \times 10^1 = 1 \label{equation1.5.4}
\end{equation}
此次回代得到计算解$x=1$,$y=1$,这与精确解的接近程度符合合理预期——计算解与精确解在3位有效数字上一致。

为什么部分主元法会产生差异?答案在于比较式\ref{equation1.5.1}、\ref{equation1.5.2}与式\ref{equation1.5.3}、\ref{equation1.5.4}

不使用部分主元法时,乘数为$10^4$,这个数值太大,完全掩盖了相对较小的数字$1$和$2$的算术运算,导致它们无法被考虑在内。也就是说,较小的数字$1$和$2$被“淹没”了,仿佛从未存在过,因此$3$位精度计算得到的是另一个方程组的精确解,该方程组与原方程组差异很大:
$$
\left(\begin{array}{cc|c}
-10^{-4} & 1 & 1 \\
1 & 0 & 0
\end{array}
\right)    
$$
使用部分主元法时,乘数为$10^{-4}$,这个数值足够小,不会淹没数字$1$和$2$。在这种情况下,$3$位精度计算得到的是方程组$\left(\begin{array}{cc|c}0 & 1 & 1 \\ 1 & 1 & 2\end{array}\right)$的精确解,该方程组与原方程组接近。

\end{example}

总之,示例\ref{example1.5.1}中的“罪魁祸首”是较大的乘数,它导致一些较小的数字无法被充分考虑,从而得到与原方程组差异很大的另一个方程组的精确解。通过在每一步最大化主元的绝对值,我们最小化了相关乘数的绝对值,从而有助于控制消元过程中出现的数字的增长,进而有助于规避舍入误差的一些影响。

当使用部分主元法时,所有乘数的绝对值都不会超过1。要理解这一点,考虑消元过程中的以下两个典型步骤:
$$
\left(\begin{array}{cccccc|c} 
* & * & * & * & * & * & * \\
0 & * & * & * & * & * & * \\
0 & 0 & \circlednum{p} & * & * & * & * \\
0 & 0 & q & * & * & * & * \\
0 & 0 & r & * & * & * & *
\end{array}\right) \xrightarrow[R_4 - (q/p)R_3]{R_5 - (r/p)R_3}
\left(
    \begin{array}{cccccc|c}
        * & * & * & * & * & * & * \\
        0 & * & * & * & * & * & * \\
        0 & 0 & \circlednum{p} & * & * & * & * \\
        0 & 0 & 0 & * & * & * & * \\
        0 & 0 & 0 & * & * & * & *
    \end{array}
\right)
$$
主元为$p$,而$q/p$和$r/p$是乘数。若采用了部分主元法,则$|p| \geq |q|$且$|p| \geq |r|$,因此:
$$
\left|\frac{q}{p}\right| \leq 1, \quad \left|\frac{r}{p}\right| \leq 1
$$
通过确保所有乘数的绝对值不超过1,产生相对较大数字(可能淹没较小数字重要性)的可能性大大降低,但这并未完全消除。要看到仍有改进空间,考虑以下示例。

\begin{example}
\label{example1.5.2}
方程组:
$$
\begin{cases} 
-10x + 10^5y = 10^5, \\
x + y = 2
\end{cases}
$$
的精确解为:
$$
x = \frac{1}{1.0001}, \quad y = \frac{1.0002}{1.0001}
$$
假设使用3位精度且采用部分主元法。由于$|-10| > 1$,无需交换行,得到:
$$
\left(\begin{array}{cc|c} 
-10 & 10^5 & 10^5 \\
1 & 1 & 2
\end{array}\right) \xrightarrow{R_2 + 10^{-1} R_1} \left(\begin{array}{cc|c} 
-10 & 10^5 & 10^5 \\
0 & 10^4 & 10^4
\end{array}\right)
$$
\end{example}
回代得到解为:
$$x=0, \quad y=1$$
这个结果必须被认为是非常糟糕的——计算出的$y$的3位解还不错,但计算出的$x$的3位解非常差!

示例\ref{example1.5.2}中的问题根源是什么?此次不能归咎于乘数。问题在于第一个方程中的系数远大于第二个方程中的系数,即由于系数的数量级不同,存在尺度问题。因此,在尝试求解方程组之前,应该以某种方式对其进行尺度变换。

若将上述示例中的第一个方程进行尺度变换,确保绝对值最大的系数为1(通过将第一个方程乘以$10^{-5}$),则得到示例\ref{example1.5.2}中的方程组,我们知道通过部分主元法可以得到与精确解非常接近的近似解。

这表明部分主元法的成功可能取决于保持系数之间的适当尺度。因此,使高斯消元法实用的第二个改进是合理的尺度变换策略。不幸的是,目前还没有已知的尺度变换方法能对所有可能的方程组都产生最优结果,因此我们必须满足于一种在大多数情况下都能工作的策略。该策略是将行尺度变换(将选定的行乘以非零乘数)与列尺度变换(将系数矩阵$A$的选定列乘以非零乘数)相结合。

行尺度变换不会改变精确解,但列尺度变换会改变精确解。列尺度变换相当于改变第$k$个未知数的单位。例如,若增广矩阵$[A|b]$中第$k$个未知数$x_k$的单位是毫米,且将$A$的第$k$列乘以$0.001$,则尺度变换后的方程组$[\hat{A}|b]$中的第$k$个未知数为$\hat{x}_k = 1000x_k$,因此尺度变换后的未知数$\hat{x}_k$的单位变为米。

经验表明,以下结合行尺度变换和列尺度变换的策略通常能很好地工作:

\begin{bluebox}{实用尺度变换策略}
\begin{itemize}
    \item 选择与问题自然相关的单位,且不扭曲事物大小之间的关系。这些自然单位通常是显而易见的,在此之后通常不再尝试进一步的列尺度变换。
    \item 对增广矩阵$[A|b]$进行行尺度变换,使得$A$的每一行中绝对值最大的系数等于1。也就是说,将每个方程除以其绝对值最大的系数。
\end{itemize}
部分主元法与上述尺度变换策略相结合,使带回代的高斯消元法成为一种极其有效的工具。长期以来,这种技术已被证明能可靠地求解实际工作中遇到的大多数线性方程组。
\end{bluebox}
尽管应用不广泛,但存在一种部分主元法的扩展形式,称为完全主元法,在某些特殊情况下,它在帮助控制舍入误差的影响方面可能比部分主元法更有效。

\begin{bluebox}{完全主元法}
若$[A|b]$是高斯消元法第$k$步的增广矩阵,则搜索主元位置以及$A$中主元位置下方和右侧的所有位置,找到绝对值最大的系数。如有必要,执行相应的行交换和列交换,将绝对值最大的系数带入主元位置。下面展示了典型情况下的第三步:
$$
\left(\begin{array}{cccccc|c} 
* & * & * & * & * & * & * \\
0 & * & * & * & * & * & * \\
0 & 0 & \circlednum{s} & S & S & * & * \\
0 & 0 & S & S & S & * & *
\end{array}\right)
$$

搜索标记为“S”的位置,找到绝对值最大的系数;如有必要,交换行和列,将该最大系数带入标记的主元位置。$A$中列交换的效果相当于对相关未知数进行置换(或重命名)。
\end{bluebox}

不难看出,完全主元法至少应与部分主元法一样有效。此外,有可能构造出特殊示例,其中完全主元法优于部分主元法。然而,在实际应用中很少遇到此类方程组。

\begin{example}
\label{example1.5.3}
问题:使用3位精度和完全主元法求解以下方程组:
$$
\begin{cases} 
    x - y = -2, \\
-9x + 10y = 12
\end{cases}
$$
解:由于10是搜索范围内绝对值最大的系数,交换第一行和第二行,然后交换第一列和第二列:
$$
\left(\begin{array}{cc|c} 
1 & -1 & -2 \\
-9 & 10 & 12 
\end{array}\right) \xrightarrow{} \left(\begin{array}{cc|c} 
-9 & 10 & 12 \\
1 & -1 & -2 
\end{array}\right) \xrightarrow{} \left(\begin{array}{cc|c} 
10 & -9 & 12 \\
-1 & 1 & -2 
\end{array}\right) \xrightarrow{} \left(\begin{array}{cc|c}
10 & -9 & 12 \\
0 & 1 & -8    
\end{array}\right)
$$
列交换的效果是将未知数重命名为$\hat{x}$和$\hat{y}$,其中$\hat{x}=y$,$\hat{y}=x$。回代得到:
$$x=\hat{y}=-8, \quad y=\hat{x}=-6$$

在这种情况下,3位解与精确解一致。若仅使用部分主元法,3位解的精度不会这么高。但如果使用带尺度变换的部分主元法,结果与使用完全主元法的结果相同。
\end{example}

如果使用完全主元法的成本与使用部分主元法的成本几乎相同,我们会始终使用完全主元法。然而,不难证明,完全主元法的成本大约是直接高斯消元法的两倍,而部分主元法仅增加微不足道的成本。再加上在实际应用中,极少遇到带尺度变换的部分主元法不足以处理但完全主元法可以处理的方程组,因此很容易理解为什么完全主元法在实践中很少使用。对于中等规模的稠密方程组(即没有大量零元素的方程组),带尺度变换的部分主元法的高斯消元法是首选方法。

\begin{exercise}
考虑下面的方程组:
$$
\begin{cases} 
10^{-3}x - y = 1, \\
x + y = 0
\end{cases}
$$
\begin{enumerate}[label=(\alph*)]
    \item 使用3位精度且不使用主元法求解该方程组。\label{item1.5.1.a}
    \item 找到一个由\ref{item1.5.1.a}中解精确满足的方程组,并指出该方程组与原方程组的接近程度。
    \item 现在使用部分主元法和3位精度求解原方程组。\label{item1.5.1.b}
    \item 找到一个由\ref{item1.5.1.b}中解精确满足的方程组,并指出该方程组与原方程组的接近程度。
    \item 使用精确算术求解原方程组,并将精确解与\ref{item1.5.1.a}和\ref{item1.5.1.b}的结果进行比较。
    \item 将精确解舍入到三位有效数字,并与\ref{item1.5.1.a}和\ref{item1.5.1.b}的结果进行比较。
\end{enumerate}
\end{exercise}

\begin{exercise}
考虑以下方程组:
$$
\begin{cases} 
    x + y = 3, \\
 -10x + 10^5y = 10^5
\end{cases}
$$
\begin{enumerate}[label=(\alph*)]
    \item 使用4位算术、部分主元法且不进行尺度变换,计算方程组的解。\label{item1.5.2.a}
    \item 使用4位算术、完全主元法且不进行尺度变换,计算原方程组的解。\label{item1.5.2.b}
    \item 此次先对原方程组进行行尺度变换,然后使用4位算术和部分主元法计算方程组的解。\label{item1.5.2.c}
    \item 现在确定方程组的精确解,并将其与\ref{item1.5.2.a}、\ref{item1.5.2.b}和\ref{item1.5.2.c}的结果进行比较。
\end{enumerate}
\end{exercise}

\begin{exercise}
在不进行尺度变换的情况下,分别使用无部分主元法和部分主元法,计算以下方程组的3位解,将结果与精确解进行比较。
$$
\begin{cases} 
-3x + y = -2, \\
10x - 3y = 7
\end{cases}
$$
\end{exercise}

\begin{exercise}
考虑以下方程组,其系数矩阵为希尔伯特矩阵:
$$
\begin{cases} 
x + \frac{1}{2}y + \frac{1}{3}z = \frac{1}{3}, \\
\frac{1}{2}x + \frac{1}{3}y + \frac{1}{4}z = \frac{1}{3}, \\
\frac{1}{3}x + \frac{1}{4}y + \frac{1}{5}z = \frac{1}{3}
\end{cases}
$$
\begin{enumerate}[label=(\alph*)]
    \item 首先将系数转换为3位浮点数,然后使用3位算术、部分主元法但不进行尺度变换,计算方程组的解。\label{item1.5.4.a}
    \item 再次使用3位算术,但先将系数(转换为浮点数后)进行行尺度变换,然后使用部分主元法计算方程组的解。\label{item1.5.4.b}
    \item 按照(b)中的步骤进行,但此次在每一步消元前都对系数进行行尺度变换。\label{item1.5.4.c}
    \item 现在使用精确算术对原方程组求解,得到精确解,并将结果与\ref{item1.5.4.a}、\ref{item1.5.4.b}和\ref{item1.5.4.c}的结果进行比较。
\end{enumerate}
\end{exercise}
